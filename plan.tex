\documentclass{article}
\title{Slut project.}
\author{Eddie Englund}

\begin{document}
    \maketitle


    \section{Beskrivning}

        Hemsidan är planerad att vara ungefär som en social media. Det ska finnas proggresift laddade bilder och videor. Hemsidan kommer att vara skriven i ramvärket \textit{React} för att använda mig utav javascript så mycket så möjligt. Dessutom så kommer hemsidan att ha en backend skriven i nodejs och använder sig utav ramvärket \textit{Express} för att bygga api'n. 
    \section{Bakgrund}
        Vi på Eddie Ltd Ab har fått en request från en klient (Andreas) att skapa någon form av hemsida där man kan ta del av bl.a bilder och videor. Hemsidan ska följa existerande regelverk.
    \section{Metod}
        Sidor: vet ej en.
        \begin{itemize}
            \item React (js framework + jsx (js med html support))
            \item HTML
            \item Scss
        \end{itemize}
    \section{Utseende och funktionalitet}
        Hemsidan ska använda sig utav en modern design stil. Den kommer att inkludera stiler som \textit{Neumorphism}. Den ska även vara kontrast anpassad för personer som är funktions hindrade.

        På hemsidan ska man kunna skapa/registrera ett konto och personalisera sitt konto. När man har ett konto så ska man kunna kommentera och updutta bilder.
    \section{Tester och anpassningar}
        Jag kommer att använda mig utav olika sätt att kontrollera att allting fungerar som det ska. Bl.a ska jag använda mig utav ett program som kallas för \textit{Postman} för att verifiera att alla routes på min api fungerar som den ska. Dessutom ska jag testa hemsidan i olika webbläsare som t.ex Safari, Firefox och Chrome. För att se till att stilen och funktionaliten är konsekvent.
    \section{Säkerhet och risker}
        Eftersom att api'n som nästan alla är helt öppen måste jag se till att requestsen är \textit{valida}. Alltså se till att datan som blir passerad är säker och följer riktlinjerna som api'n använder sig av. Dessutom måste jag se till att jag inte sparar sensetiv data på front-enden.
    \section{Tidsplan}
       


      \begin{center}
        \begin{tabular}{ |p{1cm}|p{3cm}|} 

    
            \hline
            Vecka 1 & Se att funktionalitet som att skapa användare fungerar \\
            \hline
            Vecka 2 & Utväckla front-enden och koppla apin  \\ 
             \hline
            Vecka 3 & Stabilitets testa och finslipa produkten \\ 
            \hline
           \end{tabular}
      \end{center}

\end{document}